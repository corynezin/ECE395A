\section{Introduction}
Automatic modulation classification of wireless signals has been of interest in the signal processing community because of possible applications in cognitive radio.  If a transmitter is not known beforehand, a receiver must detect the transmitted signal, demodulate it, and finally decode it in order to communicate.  Modulation identification is a necessary step in demodulation and is therefore of interest.  It is also possible that in adversarial situations, knowledge of the predominant signal modulation can aid a friendly transmitter in avoiding detection, or allow for more targeted jamming techniques against adversaries.

Convolutional neural networks have been demonstrated as superior to classic techniques in modulation classificaiton.  However, even once they are trained, neural networks are computationally intensive and thus require significant time and power to run on general purpose processors.  MAIA is a hardware architecture tailor made to solve the problem of wireless signal modulation classification using neural networks.  We designed and implemented several core components of a convolutional neural network directly in hardware, explicitly pipelining and optimizing where possible.

The core driving factors of our design are latency and size.  Our goal is to design an architecture which has very little delay between raw received IQ samples and modulation decision made, particularly it should be faster than a general purpose processor.  However, we also want to use a relatively small amount of hardware to reduce size, particularly it should be smaller than a GPU.  We chose to prototype all designs on an FPGA, as a modifiable emulation of true custom hardware (ASIC).  We are using a ZedBoard which contains a Z-7020 FPGA with 85K logic cells, 220 DSP slices, and 4.9Mb of block ram and costs about \$135.00 standalone.
